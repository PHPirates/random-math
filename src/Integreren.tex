      Onthoud:
      \begin{align*}
          \int \frac{1}{\sin^2 u} \d u = \cotan u = \frac{1}{\tan u}
      \end{align*}
\subsection{Integreren van even machten cosinus of sinus}\label{subsec:integrerenVanEvenMachtenCosinusOfSinus}
Mocht je iets willen integreren met $\cos^2$ dan volgt uit de verdubbelingsformules van sectie~\ref{gonioformules}
	 \begin{align*}
		  \cos (2\alpha) &= \cos^2 \alpha - \sin^2 \alpha \\
	 		 &=  \cos^2 \alpha + \cos^2 \alpha - \cos^2 \alpha - \sin^2 \alpha \\
	 		 &= 2\cos^2 \alpha -1
	 	 \end{align*}
	 	 en evenzo voor $\sin^2 \alpha$.
      \subsection{Partieel integreren}\label{subsec:partieelIntegreren}
          \[
              \int_a^b f(x) g'(x) \d x = \big[f(x)g(x)\big]_a^b - \int_a^b f'(x) g(x) \d x
           \]
           dus kies $f$ en $g$ zodanig dat $f'(x)g(x)$ makkelijk integreert.
           \subsubsection{Integeren van de (re\"ele) logaritme}
		We nemen $f(x)=\ln x$ en $g'(x)=1$ zodat
		\begin{align*}
			\int \ln x \cdot 1 \d x &= \int f(x)g'(x) \d x \\
			&= x \ln x - x \,.
		\end{align*}
      \subsection{Integreren door twee keer partieel}\label{subsec:integrerenDoorTweeKeerPartieel}
          Als je een integraal van bijvoorbeeld de vorm
          \[
              \int \sin t \cdot e^t \d t
          \]
          hebt, kun je opmerken dat als je een sinus twee keer integreert je weer een sinus
          krijgt, en dus weer dezelfde integraal.
          Dit kunnen we toepassen
          door twee keer partieel te integreren (grenzen weggelaten voor de duidelijkheid).
          \begin{align*}
              \int \sin t \cdot e^t \d t &= \sin t \cdot e^t - \int \cos t \cdot e^t \d t
              \quad \text{ met $f=\sin t$ en $g'=e^t$} \\
              &= \sin t \cdot e^t - \left( \cos t \cdot e^t - \int - \sin t \cdot e^t \d t \right)
              \quad \text{ met $f=\cos t$ en $g'=e^t$} \\
              &= ( \sin t - \cos t ) e^t -  \int \sin t \cdot e^t \d t
          \end{align*}
          Hieruit volgt dat
          \[  \int \sin t \cdot e^t \d t = \frac{1}{2} ( \sin t - \cos t ) e^t\,. \]
      \subsection{Integreren door breuksplitsen}\label{subsec:integrerenDoorBreuksplitsen}
          We bekijken
          \begin{align*}
              \int \frac{1}{u(1-u)} \d u &= \int \frac{A}{u} + \frac{B}{1-u} \d u \\
              &= \int \frac{A(1-u)+Bu}{u(1-u)} \d u
          \end{align*}
          waaruit volgt $A=B=1$ en dus
          \[ \int \frac{1}{u(1-u)} \d u = \ln |u| + \ln |1-u|\,. \]
      \subsection{Integreren door substitutie}\label{subsec:integrerenDoorSubstitutie}
          We bekijken
          \[ \int \sin t \cdot \cos t \cdot e^{\sin t} \d t\,. \]
          Hierin zou je een functie en zijn afgeleide kunnen herkennen, namelijk
          $\sin t$ en $\cos t$ waarbij een eventuele factor natuurlijk niet uitmaakt.
          We substitueren $u = \sin t$, waaruit volgt $\d u = \cos t \d t$
          zodat de integraal overgaat in
          \[ \int u e^u \d u \]
          wat we kunnen uitrekenen met partieel integeren.
          Er volgt
          \begin{align*}
              \int u e^u \d u &= u e^u - \int e^u \d u \\
              &=  u e^u -  e^u + c \\
              &= (\sin t -1) e^{\sin t} \,.
          \end{align*}
          Mochten er grenzen zijn, bijvoorbeeld
          \[ \int_0^{\frac{\pi}{2}} \sin t \cdot \cos t \cdot e^{\sin t} \d t\,. \]
          dan moeten we deze grenzen meesubstitueren, dus
          \[ \int_{\sin 0}^{\sin \frac{\pi}{2}} u e^u \d u = 1\,. \]
	\subsubsection{Integreren van oneven machten van de sinus}
		We bekijken
		\begin{align*}
			\int \sin^{2n+1} x \d x = \int \left( 1-\cos^2 x \right)^n \sin x \d x \,.
		\end{align*}
		We substitueren $u=\cos x$ dus $\d u = - \sin x \d x$.
		Dan wordt dit
		\begin{align*}
			\int -(1-u^2)^n \d u\,.
		\end{align*}