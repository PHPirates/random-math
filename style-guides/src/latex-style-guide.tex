%%
%% Author: Thomas
%%

% Preamble
%\batchmode
% Use exam to provide a test for thomaspackage
\documentclass[11pt]{exam}

% Packages
\usepackage[en]{../../personal-packages/thomaspackage}
\usepackage{showexpl}

% Setup
\lstset
{
    language=[LaTeX]TeX,
    breaklines=true,
    basicstyle=\tt,
    keywordstyle=\color{blue},
    identifierstyle=\color{magenta},
}

%\renewcommand{\verb}{\lstinline}

\title{\LaTeX\ style guide}
\author{Thomas Schouten \\ Abby Berkers}

% Document
\begin{document}

    \maketitle
    \begin{enumerate}
        \item Do not use \verb|\\| or \verb|\newline| outside of tabular or math environments.

        \item All sentences should start with a Capital and end with a full stop whenever reasonably possible.

        \item All sentences should start on a new line (in the source code).
        \begin{lstlisting}
This is a first sentence.
This sentence starts on a new line.
        \end{lstlisting}
        This is a first sentence.
        This sentence starts on a new line.

        \item Use \verb|$...$| for inline math, and \verb|\[...\]| for display math (on its own line).

        \item Use \verb|\left( ... \right)| to scale brackets, so the result is $\left( \frac{\pi^2}{i} \right)$ instead of $(\frac{\pi^2}{i})$.

        \item For integrals use \verb|\d x| instead of \verb|dx|, which results in $\d x$ instead of $dx$.

        \item To place headers above your text, use \verb|\section{title}|, \verb|\subsection{title}|, \verb|\subsubsection{title}| or \verb|\paragraph{title}|.

        \item Use \verb|\sin| and \verb|\cos|, so $\cos$ instead of $cos$.

        \item Use \verb|\implies| ($\implies$) and not \verb|\Rightarrow| ($\Rightarrow$), \verb|\iff| ($\iff$) and not \verb|\Leftrightarrow| ($\Leftrightarrow$).

        \item Use \verb|``...''| for ``quotes'', not \verb|"..."| for "strange quotes".

        \item For bold math in math environments (used for e.g. vectors), use \verb|$\bm{\delta}$| instead of \verb|$\textbf{\delta}$|, as        \verb|\textbf{...}| does not work on math characters like $\delta$.

        \item Use the align environment for multiple line math.
            \begin{lstlisting}
\begin{align*}
    3x &= 1 \\
    &= x_3
\end{align*}
            \end{lstlisting}

        \item Use indentation in environments, like
            \begin{lstlisting}
\begin{equation}
    \det
    \begin{pmatrix}
        1 & 2 \\
        3 & 4
    \end{pmatrix}
    = -2 \,.
\end{equation}
            \end{lstlisting}
        \item Use backticks for opening quotes, so ``quote'' and `quote'.
        \item Use non-breaking spaces in front of references and start the referring thing with a capital, so \verb|Figure~\ref{fig:myplaatje}| instead of \verb|figure \ref{myplaatje}|.
        \item For separators in filenames, references etc.\ use a hyphen/dash, so eq:power-example.
        \item In text the hyphen/dash should be typeset with \verb|--|.
        \item Regarding styling in tables:
        \begin{itemize}
            \item Try to not use vertical lines.
            \item Use the \verb|booktabs| package for better spacing, then replace \verb|\hrule| with \verb|\midrule|.
            \item Wrap the table in a figure environment.
            \item Caption should be below the table, place reference label after caption in any case.
            \item Right align numeric columns, left align text columns.
        \end{itemize}
    \end{enumerate}

\end{document}